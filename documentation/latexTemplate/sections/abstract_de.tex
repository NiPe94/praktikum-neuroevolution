%% LaTeX2e class for student theses
%% sections/abstract_de.tex
%% 
%% Karlsruhe Institute of Technology
%% Institute for Program Structures and Data Organization
%% Chair for Software Design and Quality (SDQ)
%%
%% Dr.-Ing. Erik Burger
%% burger@kit.edu
%%
%% Version 1.3.3, 2018-04-17

\Abstract
In dieser Arbeit wird ein Ansatz vorgestellt, um Modelle der künstlichen Intelligenz für das Spielen von Atari-Spielen und dem Lösen von Robotik-Problemen zu trainieren. Die Grundlage dafür bildet der Algorithmus Neuroevolution of Augmenting Topologies (NEAT), ein Vertreter der evolutionen Algorithmen. Eine Implementierung des Algorithmus findet sich unter Anderem mit NEAT-Python, welches zusammen mit dem OpenAI Gym Framework die Implementierung eines Trainingsansatzes für Atari-Spiele und Robotik-Probleme ermöglicht. Die Ergebnisse zeigen auf, dass für das Lernen von Atari-Spielen eine Bildvorverarbeitung sinnvoll ist und NEAT für Robotik-Probleme im Rahmen dieser Arbeit als ungeeignet erscheint.
%% LaTeX2e class for student theses
%% sections/content.tex
%% 
%% Karlsruhe Institute of Technology
%% Institute for Program Structures and Data Organization
%% Chair for Software Design and Quality (SDQ)
%%
%% Dr.-Ing. Erik Burger
%% burger@kit.edu
%%
%% Version 1.3.3, 2018-04-17

\chapter{Einleitung}
\label{ch:Introduction}
Die Menschheit hat sich im Laufe der Zeit über Generationen hinweg weiterentwickelt. Jede Generation lernt, mit der gegebenen Umwelt besser klarzukommen, als die Vorherige. Nur die Individuen, die in der frühen Vergangenheit nicht gegessen wurden oder andersweitig versagten, sind weitergekommen und gaben ihr gelerntes Wissen an die Nachkommen weiter. Diese Art des Lernens findet auch im Bereich der künstlichen Intelligenz ihre Anwendung. Sogenannte evolutionäre Algorithmen erlauben das Trainieren von ganzen Gruppen aus künstlichen Intelligenzen, die gegeneinander um den Platz des Besten konkurrieren.
\\
\\
Um einen Konkurrenzkampf zu ermöglichen, ist die Definition einer Herausforderung erforderlich. Die Gestalt solcher zu lösender Probleme ist sehr vielfältig. Es wurden bereits Veröffentlichung gemacht, in denen künstliche Intelligenzen versuchen, verschiedene Videospiele zu meistern. Dazu gehören zum Beispiel Atari- \cite{stanley2002} oder Sega-Spiele \cite{gupta2019}. Weiterhin gibt es Situationen, in denen mithilfe verschiedener Arten von Robotern Versuche unternommen werden, sich möglichst schnell fortzubewegen. Die Schwierigkeit liegt hierbei darin, alle gegebenen Gelenke eines Roboters optimal anzusteuern, um das definierte Ziel zu erreichen.
\\
\\
In dieser Arbeit sollen künstliche Intelligenzen für das Spielen von Atari-Spielen und dem Lösen von Robotik-Problemen trainiert werden. Dafür gestaltet sich der Inhalt dieser Arbeit folgendermaßen. Zu Beginn werden die essentiellen Grundlagen für das Problemverstehen beleuchtet. Dazu gehört das Thema der evolutionären Algorithmen und ein Vertreter dieser Gruppe namens Neuroevolution of Augmenting Topologies (NEAT). Außerdem werden das Framework Gym von OpenAI zur Bereitstellung von Trainingsumgebungen und eine Auswahl an NEAT-Implementierungen vorgestellt. Danach werden die vorgestellten Frameworks in einem Implementierungsansatz kombiniert, um sowohl das Training von Atari-Spielen als auch Robotik-Problemen zu ermöglichen. Im Anschluss daran werden die Trainingserfahrungen und Ergebnisse präsentert, ehe die Arbeit mit einer Zusammenfassung und einem Ausblick endet.
